\documentclass[american]{scrartcl}

\usepackage{babel}
\usepackage{xcolor}

\usepackage{mkessler-math}
\usepackage{mkessler-vocab}
\usepackage{mkessler-enumerate}
\usepackage{fancythm}
\usepackage{csquotes}

\title{Noncommutative Spaces}
\subtitle{Lecture Course, Summer Semester 2025}
\author{Koen van den Dungen\thanks{Notes by Melvin Weiß}}


\newcommand{\todo}[1]{\textcolor{red}{TODO: #1}}
\newcommand{\weaktodo}[1]{\textcolor{orange}{Maybe: #1}}
\renewcommand{\norm}[1]{\left\lVert #1 \right\rVert}
\newcommand{\cB}{\mathcal{B}}
\newcommand{\cC}{\mathcal{C}}
\newcommand{\inv}{^{-1}}
\newcommand{\eps}{\varepsilon}
\DeclareSimpleMathOperator{const}
\let\sp\relax
\DeclareSimpleMathOperator{sp}
\DeclareSimpleMathOperator{mSpec}
\DeclareSimpleMathOperator{ev}
\begin{document}
	\maketitle
	
	\cleardoublepage
	
	\tableofcontents
	
	\cleardoublepage
	
	
	
	\section{Introduction}
	\todo{Motivation}
	
	\section{$C^*$-Algebras}
	\subsection{Banach algebras}
	\begin{definition}
		A \textit{Banach algebra} is a (not necessarily unital or commutative) $\C$-algebra $A$ together with a norm $\norm{.}:A\to \R$ such that:
		\begin{itemize}
			\item $\norm{.}$ is \textit{submultiplicative}: $\norm{ab}\leq\norm{a}\norm{b}$ for all $a,b\in A$.
			\item $(A,\norm{.})$ is a \textit{Banach space}: A complete normed vector space.
		\end{itemize}
	\end{definition}
	\begin{remark}
		The multiplication on a Banach algebra $A$ is continuous: As for all $a,b\in A$ we have $\norm{ab}\leq \norm{a}\norm{b}$, the linear map $a\cdot(-)\colon A\to A$ is a bounded operator, hence continuous.
	\end{remark}
	\begin{remark}
		We can usually assume $A$ to be \textit{unital} (i.e. there is some $1\in A$ with $1\cdot a=a\cdot 1 = a$ for all $a\in A$), otherwise replacing it by the \textit{unitization} $\tilde{A}$ of $A$, given by:
		\begin{align*}
			\tilde{A}\coloneqq A\oplus\C
		\end{align*}
		with the multiplication
		\begin{align*}
			(a,\lambda)\cdot (b,\mu)\coloneqq (ab+\lambda B+\mu A,\lambda\mu)
		\end{align*}
		and the norm
		\begin{align*}
			\norm{(a,\lambda)}\coloneqq \norm{a}+\abs{\lambda}.
		\end{align*}
		This is in fact a unital Banach algebra: The unit is given by $(0,1)$, as witnessed by
		\begin{align*}
			(0,1)\cdot (a,\lambda)=(a,\lambda)=(a,\lambda)\cdot (0,1)
		\end{align*}
		for $(a,\lambda)\in \tilde{A}$. $\tilde{A}$ is a Banach space as $\C$ is one and the sum of Banach spaces is again a Banach space. Submultiplicativity follows from
		\begin{align*}
			\norm{(a,\lambda)\cdot (b,\mu)}&=\norm{ab+\lambda b+\mu a}+\abs{\lambda\mu}\\
			&\leq \norm{ab}+\norm{\lambda b}+\norm{\mu a}+\abs{\lambda\mu}\\
			&\leq\norm{a}\norm{b}+\abs{\lambda} \norm{b}+\abs{\mu}\norm{a}+\abs{\lambda}\abs{\mu}\\
			&=\norm{(a,\lambda)}\norm{(b,\mu)}.
		\end{align*}
		 Confirming the algebra structure is a straightforward check.
		\weaktodo{Remark on adjunction}
	\end{remark}
	\begin{example}
		\begin{enumerate}
			\item Let $V$ be a Banach space. Then
			\begin{align*}
				\cB(V)\coloneqq \left\{T\colon V\to V\mid T\text{ bounded linear}\right\}
			\end{align*}
			with norm $\norm{T}\coloneqq\sup_{v\in V}\frac{\norm{Tv}}{\norm{v}}$ and composition as multiplication is a unital Banach algebra.
			\item Let $X$ be a topological space. We can define
			\begin{align*}
				\cC_b(X)\coloneqq \left\{f:X\to \C\mid f\text{ continuous, }\sup_{x\in X}\abs{f(x)}<\infty\right\}
			\end{align*}
			and
			\begin{align*}
				\cC_0(X)\coloneqq\left\{f\in \cC_b(X)\mid\forall \eps>0\exists K\subseteq X \text{ compact, }f\inv((-\eps,\eps))\subseteq K\right\}
			\end{align*}
			with pointwise multiplication and $\norm{f}\coloneqq\sup_{x\in X}\abs{f(x)}$. Both of these form Banach algebras. $\cC_b$ is always unital with unit $\const_1$, whereas $\cC_0$ is unital if and only if $X$ is compact.
		\end{enumerate}
	\end{example}
	\begin{definition}
		A \textit{(twosided) ideal} $J\subseteq A$ is a subspace $J\subseteq A$ with $AJ\subseteq J$ and $JA\subseteq J$. This is equivalent to $J$ being a twosided ideal of $A$ viewed as an ordinary (non-unital) ring.
	\end{definition}
	\begin{lemma}\label{Lemma: Quotient Banach algebra}
		If $J\subseteq A$ is a closed ideal, the quotient ring $A/J$ equipped with the norm
		\begin{align*}
			\norm{a+J}\coloneqq \inf_{j\in J}\norm{a+j}
		\end{align*}
		is again a Banach algebra.
	\end{lemma}
	\begin{proof}
		Quotients of algebras under twosided ideals are again algebras, hence so is $A/J$. Further, the underlying normed vector space of $A/J$ agrees with the quotient $A/J$ of underlying normed vector spaces, hence is the quotient of a Banach space by a closed subspace and as such again a Banach space. Lastly, as $J$ is an ideal, for $a,b\in A$ and $j,k\in J$ we have $aj+bk+jk\in J$, hence
		\begin{align*}
			\norm{(a+J)(b+J)}&=\inf_{j\in J}\norm{ab+j}\\
			&\leq \inf_{j,k\in J}\norm{ab+aj+bk+jk}\\
			&=\inf_{j,k\in J} \norm{(a+j)(b+k)}\\
			&\leq \inf_{j,k\in J} \norm{a+j}\norm{b+k}\\
			&=\left(\inf_{j\in J} \norm{a+j}\right)\left(\inf_{k\in J} \norm{b+k}\right)\\
			&=\norm{a+J}\norm{b+J},
		\end{align*}
		so submultiplicativity holds.
	\end{proof}
	\begin{example}
		For a Banach algebra $A$, $A\subseteq \tilde{A}$ is a twosided ideal.
	\end{example}
	\begin{proof}
		The map $p:\tilde{A}\to \C, (a,\lambda)\mapsto \lambda$ is a ring homomorphism, hence the kernel $\ker p=A$ is a twosided ideal. Further, $p$ is continuous and $\{0\}\subseteq\C$ is closed, hence so is $A$.
	\end{proof}
	\begin{definition}
		For a unital Banach algebra $A$ and an element $a\in A$, we define the \textit{spectrum}
		\begin{align*}
			\sp(a)\coloneqq \{\lambda\in \C\mid (\lambda\cdot 1-a)\not\in A^{\times}\},
		\end{align*}
		where $A^{\times}$ is the group of units of $A$. We further define the spectral radius
		\begin{align*}
			r(a)\coloneqq \sup_{\lambda\in \sp(a)}\abs{\lambda}.
		\end{align*}
	\end{definition}
	This spectrum for elements of a general Banach algebra behaves just like the familiar one:
	\begin{theorem}\label{Thm: Spectral Theorem}
		Let $A$ be a unital Banach algebra and $a\in A$. Then:
		\begin{itemize}
			\item $\sp(a)\subseteq \C$ is non-empty and compact.
			\item The following formula describes the spectral radius:
			\begin{align*}
				r(a)=\lim_{n\to\infty} \norm{a^n}^{\frac{1}{n}}.
			\end{align*}
			Especially, by submultiplicativity, $r(a)\leq \norm{a}$.
		\end{itemize}
	\end{theorem}
	\begin{corollary}[Gelfand-Mazur]\label{Cor: Gelfand-Mazur}
		Let $A$ be a unital Banach algebra with $A^{\times}=A\backslash\{0\}$. Then $A\cong \C$.
	\end{corollary}
	\begin{proof}[Proof sketch]
		Let $a\in A^{\times}$. By Theorem \ref{Thm: Spectral Theorem}, there is some $\lambda\in \C$ with $\lambda\cdot 1-a\in A\backslash A^{\times}=\{0\}$, so $a=\lambda\cdot 1$. This provides an isomorphism $A\cong \C$. \weaktodo{More detail.}
	\end{proof}
	\subsection{Commutative Banach algebras}
	For this section, fix a commutative Banach algebra $A$.
	\begin{definition}
		\leavevmode
		\begin{itemize}
			\item A \textit{character} on $A$ is a non-zero $\C$-algebra homomorphism $\chi:A\to\C$.
			\item The \textit{spectrum} of $A$ is
			\begin{align*}
				\hat{A}\coloneqq \left\{\chi:A\to \C\mid\chi\text{ character}\right\}
			\end{align*}
		\end{itemize}
	\end{definition}
	\begin{example}
		As we will see in \todo{reference}, we have:
		\begin{itemize}
			\item For a locally compact Hausdorff space $X$,
			\begin{align*}
				X\cong \hat{\cC_0(X)}
			\end{align*}
			\item For a unital $C^*$-algebra $A$ and $a\in A$ with $aa^*=a^*a$ we have
			\begin{align*}
				\sp(a)\cong \hat{\langle 1,a\rangle},
			\end{align*}
			where $\langle 1,a\rangle\subseteq A$ is the sub-$C^*$-algebra of $A$ generated by $1$ and $a$.
		\end{itemize}
	\end{example}
	\begin{fact}\label{Fact: Maximal ideals in commutative Banach algebra}
		For a commutative Banach algebra $A$, the following hold:
		\begin{itemize}
			\item An ideal $m\subseteq A$ is maximal if and only if it has codimension 1.
			\item Maximal ideals are closed.
		\end{itemize}
	\end{fact}

	\begin{proposition}\label{Prop: Spectrum is maximal ideals}
		If additionally $A$ is unital, the map
		\begin{align*}
			\hat{A}&\to \mSpec(A)\\
			\chi &\mapsto \ker \chi,
		\end{align*}
		where $\mSpec A$ denotes the set of maximal ideals of $A$, is a bijection.
	\end{proposition}
\begin{proof}
	As $\ker \chi$ has codimension 1 and is closed, it is maximal by Fact \ref{Fact: Maximal ideals in commutative Banach algebra}, hence the map is well defined.
	
	Let $m\in\mSpec(A)$. $m$ is closed by Fact \ref{Fact: Maximal ideals in commutative Banach algebra}. Thus, $A/m$ is a Banach algebra by Lemma \ref{Lemma: Quotient Banach algebra}, and a field by maximality of $m$, hence we have $A/m\cong \C$ by Corollary \ref{Cor: Gelfand-Mazur}. But then, the quotient map $A\to A/m\cong \C$ is a character with kernel $m$, hence surjectivity.
	
	For injectivity, let $\chi,\chi'$ be characters with $\ker \chi = \ker \chi'\eqcolon m$. Note that $A/m\cong \C$ by the argument above and fix such an isomorphism. Especially, both $\chi$ and $\chi'$ factor as a composition $A\to A/m\cong \C \to \C$, where the first map is the quotient map and the second map is the fixed isomorphism. But now, as any non-zero homomorphism of $\C$-algebras from $\C$ to $\C$ is the identity\footnote{To see this, first note that any such homomorphism $\phi$ is already unital. Then, $\phi(\lambda)=\lambda\phi(1)=\lambda$ for all $\lambda\in \C$.}, the composition above is unique and $\chi=\chi'$.
\end{proof}
\weaktodo{Something about the topology on the spectrum}
\begin{lemma}\label{Lemma: Spectrum subset of unit ball in dual space}
	For $\chi \in \hat{A}$, we have $\norm{\chi}\leq 1$. In particular, $\hat{A}\subseteq A^*\coloneqq \cB(A,\C)$, where $\cB(V,W)$ denotes the Banach space of bounded linear maps $V\to W$, equipped with the operator norm.
\end{lemma}
\begin{proof}
	Suppose $A$ is unital and let $\chi\in \hat{A}$. By the proof of Proposition \ref{Prop: Spectrum is maximal ideals}, $\chi$ is already isomorphic to the quotient map $p:A\to A/(\ker\chi)$. But that quotient map satisfies $\norm{p(a)}=\norm{a+\ker\chi}\leq \norm{a}$ for all $a\in A$, hence so does $\chi$.\\
	Now, consider the case where $A$ is not unital. We can define
	\begin{align*}
		\tilde{\chi}:\tilde{A}&\to \C\\
		(a,\lambda)&\mapsto \chi(a)+\lambda
	\end{align*}
	It is easy to see that this is again an algebra homomorphism. But then, by the unital case:
	\begin{align*}
		\abs{\chi(a)}=\abs{\tilde{\chi}((a,0))}\leq \norm{(a,0)}=\norm{a}.
	\end{align*}
	Hence, $\norm{\chi}\leq 1$ and especially $\chi$ is continuous.
\end{proof}

\begin{definition}
	We equip $A^*=\cB(A,\C)$ with the \textit{weak-$*$} topology: The coarsest topology such that for all $a\in A$
	\begin{align*}
		\ev_a:A^*&\to \C\\
		\phi &\mapsto \phi(a)
	\end{align*}
	is continuous. In other words, $\phi_n\to \phi$ in $A^*$ if and only if $\phi_n(a)\to \phi(a)$ for all $a\in A$.
	We further equip $\hat{A}\subseteq A^*$ with the subspace topology.
\end{definition}
\begin{lemma}\label{Lemma: Dual space Hausdorff}
	$A^*$ is Hausdorff. In particular, so is $\hat{A}$.
\end{lemma}
\begin{proof}
	Let $\phi,\phi'\in A^*$. Then there is some $a\in A$ with $\phi(a)\neq \phi'(a)$. Now, picking disjoint opens $U, V\subseteq \C$ with $\phi(a)\in U$, $\phi'(a)\in V$, their preimages form disjoint opens $\ev_a\inv(U), \ev_a\inv(V)\subseteq A^*$ with $\phi\in \ev_a\inv(U)$ and $\phi'\in \ev_a\inv(V)$.
\end{proof}
\begin{theorem}\label{Thm: Banach-Alaoglu}[Banach-Alaoglu]
	The closed unit ball $B^*\coloneqq \{\phi\in A^*\mid \norm{\phi}\leq 1\}$ is compact.
\end{theorem}
\begin{proof}
	Omitted.
\end{proof}
\begin{proposition}
	$\hat{A}$ is locally compact. If $A$ is unital, $\hat{A}$ is compact.
\end{proposition}
\begin{proof}
	By Lemma \ref{Lemma: Spectrum subset of unit ball in dual space}, $\hat{A}$ is a subspace of the closed unit ball $B^*\coloneqq \{\phi\in A^*\mid \norm{\phi}\leq 1\}$, which is compact by Theorem \ref{Thm: Banach-Alaoglu} and Hausdorff by Lemma \ref{Lemma: Dual space Hausdorff}, hence for the first part of the statement it suffices to show that $\hat{A}\cup \{0\}\subseteq B^*$ is closed. Pick a sequence $(\phi_n)_{n\in \N}\in (\hat{A}\cup\{0\})^{\N}$.
\end{proof}
\end{document}
