\documentclass[american]{scrartcl}

\usepackage{babel}
\usepackage{xcolor}

\usepackage{mkessler-math}
\usepackage{mkessler-vocab}
\usepackage{mkessler-enumerate}
\usepackage{fancythm}
\usepackage{csquotes}

\title{Noncommutative Spaces}
\subtitle{Lecture Course, Summer Semester 2025}
\author{Koen van den Dungen\thanks{Notes by Melvin Weiß}}


\newcommand{\todo}[1]{\textcolor{red}{TODO: #1}}
\newcommand{\weaktodo}[1]{\textcolor{orange}{Maybe: #1}}
\renewcommand{\norm}[1]{\left\lVert #1 \right\rVert}
\newcommand{\cB}{\mathcal{B}}
\newcommand{\cC}{\mathcal{C}}
\newcommand{\cH}{\mathcal{H}}
\newcommand{\inv}{^{-1}}
\newcommand{\eps}{\varepsilon}
\DeclareSimpleMathOperator{const}
\let\sp\relax
\DeclareSimpleMathOperator{sp}
\DeclareSimpleMathOperator{mSpec}
\DeclareSimpleMathOperator{ev}
\newcommand{\conj}[1]{\overline{#1}}
\DeclareMathOperator{\oplow}{op}
\renewcommand{\op}{^{\oplow}}
\newcommand{\into}{\hookrightarrow}
\DeclareSimpleMathOperator{comm}
\begin{document}
	\maketitle
	
	\cleardoublepage
	
	\tableofcontents
	
	\cleardoublepage
	
	
	
	\section{Introduction}
	\todo{Motivation}
	
	\section{$C^*$-Algebras}
	\subsection{Banach algebras}
	\begin{definition}
		A \textit{Banach algebra} is a (not necessarily unital or commutative) $\C$-algebra $A$ together with a norm $\norm{.}:A\to \R$ such that:
		\begin{itemize}
			\item $\norm{.}$ is \textit{submultiplicative}: $\norm{ab}\leq\norm{a}\norm{b}$ for all $a,b\in A$.
			\item $(A,\norm{.})$ is a \textit{Banach space}: A complete normed vector space.
		\end{itemize}
	\end{definition}
	\begin{remark}
		The multiplication on a Banach algebra $A$ is continuous: As for all $a,b\in A$ we have $\norm{ab}\leq \norm{a}\norm{b}$, the linear map $a\cdot(-)\colon A\to A$ is a bounded operator, hence continuous.
	\end{remark}
	\begin{remark}
		We can usually assume $A$ to be \textit{unital} (i.e. there is some $1\in A$ with $1\cdot a=a\cdot 1 = a$ for all $a\in A$), otherwise replacing it by the \textit{unitization} $\tilde{A}$ of $A$, given by:
		\begin{align*}
			\tilde{A}\coloneqq A\oplus\C
		\end{align*}
		with the multiplication
		\begin{align*}
			(a,\lambda)\cdot (b,\mu)\coloneqq (ab+\lambda B+\mu A,\lambda\mu)
		\end{align*}
		and the norm
		\begin{align*}
			\norm{(a,\lambda)}\coloneqq \norm{a}+\abs{\lambda}.
		\end{align*}
		This is in fact a unital Banach algebra: The unit is given by $(0,1)$, as witnessed by
		\begin{align*}
			(0,1)\cdot (a,\lambda)=(a,\lambda)=(a,\lambda)\cdot (0,1)
		\end{align*}
		for $(a,\lambda)\in \tilde{A}$. $\tilde{A}$ is a Banach space as $\C$ is one and the sum of Banach spaces is again a Banach space. Submultiplicativity follows from
		\begin{align*}
			\norm{(a,\lambda)\cdot (b,\mu)}&=\norm{ab+\lambda b+\mu a}+\abs{\lambda\mu}\\
			&\leq \norm{ab}+\norm{\lambda b}+\norm{\mu a}+\abs{\lambda\mu}\\
			&\leq\norm{a}\norm{b}+\abs{\lambda} \norm{b}+\abs{\mu}\norm{a}+\abs{\lambda}\abs{\mu}\\
			&=\norm{(a,\lambda)}\norm{(b,\mu)}.
		\end{align*}
		 Confirming the algebra structure is a straightforward check.
		\weaktodo{Remark on adjunction}
	\end{remark}
	\begin{example}
		\begin{enumerate}
			\item Let $V$ be a Banach space. Then
			\begin{align*}
				\cB(V)\coloneqq \left\{T\colon V\to V\mid T\text{ bounded linear}\right\}
			\end{align*}
			with norm $\norm{T}\coloneqq\sup_{v\in V}\frac{\norm{Tv}}{\norm{v}}$ and composition as multiplication is a unital Banach algebra.
			\item Let $X$ be a topological space. We can define
			\begin{align*}
				\cC_b(X)\coloneqq \left\{f:X\to \C\mid f\text{ continuous, }\sup_{x\in X}\abs{f(x)}<\infty\right\}
			\end{align*}
			and
			\begin{align*}
				\cC_0(X)\coloneqq\left\{f\in \cC_b(X)\mid\forall \eps>0\exists K\subseteq X \text{ compact, }f\inv((-\eps,\eps))\subseteq K\right\}
			\end{align*}
			with pointwise multiplication and $\norm{f}\coloneqq\sup_{x\in X}\abs{f(x)}$. Both of these form Banach algebras. $\cC_b$ is always unital with unit $\const_1$, whereas $\cC_0$ is unital if and only if $X$ is compact.
		\end{enumerate}
	\end{example}
	\begin{definition}
		A \textit{(twosided) ideal} $J\subseteq A$ is a subspace $J\subseteq A$ with $AJ\subseteq J$ and $JA\subseteq J$. This is equivalent to $J$ being a twosided ideal of $A$ viewed as an ordinary (non-unital) ring.
	\end{definition}
	\begin{lemma}\label{Lemma: Quotient Banach algebra}
		If $J\subseteq A$ is a closed ideal, the quotient ring $A/J$ equipped with the norm
		\begin{align*}
			\norm{a+J}\coloneqq \inf_{j\in J}\norm{a+j}
		\end{align*}
		is again a Banach algebra.
	\end{lemma}
	\begin{proof}
		Quotients of algebras under twosided ideals are again algebras, hence so is $A/J$. Further, the underlying normed vector space of $A/J$ agrees with the quotient $A/J$ of underlying normed vector spaces, hence is the quotient of a Banach space by a closed subspace and as such again a Banach space. Lastly, as $J$ is an ideal, for $a,b\in A$ and $j,k\in J$ we have $aj+bk+jk\in J$, hence
		\begin{align*}
			\norm{(a+J)(b+J)}&=\inf_{j\in J}\norm{ab+j}\\
			&\leq \inf_{j,k\in J}\norm{ab+aj+bk+jk}\\
			&=\inf_{j,k\in J} \norm{(a+j)(b+k)}\\
			&\leq \inf_{j,k\in J} \norm{a+j}\norm{b+k}\\
			&=\left(\inf_{j\in J} \norm{a+j}\right)\left(\inf_{k\in J} \norm{b+k}\right)\\
			&=\norm{a+J}\norm{b+J},
		\end{align*}
		so submultiplicativity holds.
	\end{proof}
	\begin{example}
		For a Banach algebra $A$, $A\subseteq \tilde{A}$ is a twosided ideal.
	\end{example}
	\begin{proof}
		The map $p:\tilde{A}\to \C, (a,\lambda)\mapsto \lambda$ is a ring homomorphism, hence the kernel $\ker p=A$ is a twosided ideal. Further, $p$ is continuous and $\{0\}\subseteq\C$ is closed, hence so is $A$.
	\end{proof}
	\begin{definition}
		For a unital Banach algebra $A$ and an element $a\in A$, we define the \textit{spectrum}
		\begin{align*}
			\sp(a)\coloneqq \{\lambda\in \C\mid (\lambda\cdot 1-a)\not\in A^{\times}\},
		\end{align*}
		where $A^{\times}$ is the group of units of $A$. We further define the spectral radius
		\begin{align*}
			r(a)\coloneqq \sup_{\lambda\in \sp(a)}\abs{\lambda}.
		\end{align*}
	\end{definition}
	This spectrum for elements of a general Banach algebra behaves just like the familiar one:
	\begin{theorem}\label{Thm: Spectral Theorem}
		Let $A$ be a unital Banach algebra and $a\in A$. Then:
		\begin{itemize}
			\item $\sp(a)\subseteq \C$ is non-empty and compact.
			\item The following formula describes the spectral radius:
			\begin{align*}
				r(a)=\lim_{n\to\infty} \norm{a^n}^{\frac{1}{n}}.
			\end{align*}
			Especially, by submultiplicativity, $r(a)\leq \norm{a}$.
		\end{itemize}
	\end{theorem}
	\begin{corollary}[Gelfand-Mazur]\label{Cor: Gelfand-Mazur}
		Let $A$ be a unital Banach algebra with $A^{\times}=A\backslash\{0\}$. Then $A\cong \C$.
	\end{corollary}
	\begin{proof}[Proof sketch]
		Let $a\in A^{\times}$. By Theorem \ref{Thm: Spectral Theorem}, there is some $\lambda\in \C$ with $\lambda\cdot 1-a\in A\backslash A^{\times}=\{0\}$, so $a=\lambda\cdot 1$. This provides an isomorphism $A\cong \C$. \weaktodo{More detail.}
	\end{proof}
	\subsection{Commutative Banach algebras}
	For this section, fix a commutative Banach algebra $A$.
	\begin{definition}
		\leavevmode
		\begin{itemize}
			\item A \textit{character} on $A$ is a non-zero $\C$-algebra homomorphism $\chi:A\to\C$.
			\item The \textit{spectrum} of $A$ is
			\begin{align*}
				\hat{A}\coloneqq \left\{\chi:A\to \C\mid\chi\text{ character}\right\}
			\end{align*}
		\end{itemize}
	\end{definition}
	\begin{remark}\label{Rmk: Characters unital}
		If $A$ is unital, for an algebra homomorphism $\chi:A\to\C$ to be non-zero is equivalent to being unital, i.e. satisfying $\chi(1)=1$. This is because $\chi(a)=\chi(1)\chi(a)$ for all $a\in A$.
	\end{remark}
	\begin{example}
		As we will see in \todo{reference}, we have:
		\begin{itemize}
			\item For a locally compact Hausdorff space $X$,
			\begin{align*}
				X\cong \hat{\cC_0(X)}
			\end{align*}
			\item For a unital $C^*$-algebra $A$ and $a\in A$ with $aa^*=a^*a$ we have
			\begin{align*}
				\sp(a)\cong \hat{\langle 1,a\rangle},
			\end{align*}
			where $\langle 1,a\rangle\subseteq A$ is the sub-$C^*$-algebra of $A$ generated by $1$ and $a$.
		\end{itemize}
	\end{example}
	\begin{fact}\label{Fact: Maximal ideals in commutative Banach algebra}
		For a commutative Banach algebra $A$, the following hold:
		\begin{itemize}
			\item An ideal $m\subseteq A$ is maximal if and only if it has codimension 1.
			\item Maximal ideals are closed.
		\end{itemize}
	\end{fact}

	\begin{proposition}\label{Prop: Spectrum is maximal ideals}
		If additionally $A$ is unital, the map
		\begin{align*}
			\hat{A}&\to \mSpec(A)\\
			\chi &\mapsto \ker \chi,
		\end{align*}
		where $\mSpec A$ denotes the set of maximal ideals of $A$, is a bijection.
	\end{proposition}
\begin{proof}
	As $\ker \chi$ has codimension 1 and is closed, it is maximal by Fact \ref{Fact: Maximal ideals in commutative Banach algebra}, hence the map is well defined.
	
	Let $m\in\mSpec(A)$. $m$ is closed by Fact \ref{Fact: Maximal ideals in commutative Banach algebra}. Thus, $A/m$ is a Banach algebra by Lemma \ref{Lemma: Quotient Banach algebra}, and a field by maximality of $m$, hence we have $A/m\cong \C$ by Corollary \ref{Cor: Gelfand-Mazur}. But then, the quotient map $A\to A/m\cong \C$ is a character with kernel $m$, hence surjectivity.
	
	For injectivity, let $\chi,\chi'$ be characters with $\ker \chi = \ker \chi'\eqcolon m$. Note that $A/m\cong \C$ by the argument above and fix such an isomorphism. Especially, both $\chi$ and $\chi'$ factor as a composition $A\to A/m\cong \C \to \C$, where the first map is the quotient map and the second map is the fixed isomorphism. But now, as any non-zero homomorphism of $\C$-algebras from $\C$ to $\C$ is the identity\footnote{To see this, first note that any such homomorphism $\phi$ is already unital. Then, $\phi(\lambda)=\lambda\phi(1)=\lambda$ for all $\lambda\in \C$.}, the composition above is unique and $\chi=\chi'$.
\end{proof}
\weaktodo{Something about the topology on the spectrum}
\begin{lemma}\label{Lemma: Spectrum subset of unit ball in dual space}
	For $\chi \in \hat{A}$, we have $\norm{\chi}\leq 1$. In particular, $\hat{A}\subseteq A^*\coloneqq \cB(A,\C)$, where $\cB(V,W)$ denotes the Banach space of bounded linear maps $V\to W$, equipped with the operator norm.
\end{lemma}
\begin{proof}
	Suppose $A$ is unital and let $\chi\in \hat{A}$. By the proof of Proposition \ref{Prop: Spectrum is maximal ideals}, $\chi$ is already isomorphic to the quotient map $p:A\to A/(\ker\chi)$. But that quotient map satisfies $\norm{p(a)}=\norm{a+\ker\chi}\leq \norm{a}$ for all $a\in A$, hence so does $\chi$.\\
	Now, consider the case where $A$ is not unital. We can define
	\begin{align*}
		\tilde{\chi}:\tilde{A}&\to \C\\
		(a,\lambda)&\mapsto \chi(a)+\lambda
	\end{align*}
	It is easy to see that this is again an algebra homomorphism. But then, by the unital case:
	\begin{align*}
		\abs{\chi(a)}=\abs{\tilde{\chi}((a,0))}\leq \norm{(a,0)}=\norm{a}.
	\end{align*}
	Hence, $\norm{\chi}\leq 1$ and especially $\chi$ is continuous.
\end{proof}

\begin{definition}
	We equip $A^*=\cB(A,\C)$ with the \textit{weak-$*$} topology: The coarsest topology such that for all $a\in A$
	\begin{align*}
		\ev_a:A^*&\to \C\\
		\phi &\mapsto \phi(a)
	\end{align*}
	is continuous. In other words, $\phi_n\to \phi$ in $A^*$ if and only if $\phi_n(a)\to \phi(a)$ for all $a\in A$.
	We further equip $\hat{A}\subseteq A^*$ with the subspace topology.
\end{definition}
\begin{lemma}\label{Lemma: Dual space Hausdorff}
	$A^*$ is Hausdorff. In particular, so is $\hat{A}$.
\end{lemma}
\begin{proof}
	Let $\phi,\phi'\in A^*$. Then there is some $a\in A$ with $\phi(a)\neq \phi'(a)$. Now, picking disjoint opens $U, V\subseteq \C$ with $\phi(a)\in U$, $\phi'(a)\in V$, their preimages form disjoint opens $\ev_a\inv(U), \ev_a\inv(V)\subseteq A^*$ with $\phi\in \ev_a\inv(U)$ and $\phi'\in \ev_a\inv(V)$.
\end{proof}
\begin{theorem}[Banach-Alaoglu]\label{Thm: Banach-Alaoglu}
	The closed unit ball $B^*\coloneqq \{\phi\in A^*\mid \norm{\phi}\leq 1\}$ is compact.
\end{theorem}
\begin{proof}
	Omitted.
\end{proof}
\begin{proposition}
	$\hat{A}$ is locally compact. If $A$ is unital, $\hat{A}$ is compact.
\end{proposition}
\begin{proof}
	By Lemma \ref{Lemma: Spectrum subset of unit ball in dual space}, $\hat{A}$ is a subspace of the closed unit ball $B^*\coloneqq \{\phi\in A^*\mid \norm{\phi}\leq 1\}$, which is compact by Theorem \ref{Thm: Banach-Alaoglu} and Hausdorff by Lemma \ref{Lemma: Dual space Hausdorff}, hence for the first part of the statement it suffices to show that $\hat{A}\cup \{0\}\subseteq B^*$ is closed. Pick a sequence $(\phi_n)_{n\in \N}\in (\hat{A}\cup\{0\})^{\N}$ converging to $\phi\in B^*$. Then
	\begin{align*}
		\phi(ab)=\lim_{n\to\infty}\phi_n(ab)=\lim_{n\to\infty}\phi_n(a)\lim_{n\to\infty}\phi_n(b)=\phi(a)\phi(b)
	\end{align*}
	and analogous arguments for $\C$-linearity work, hence $\phi$ is again an algebra homomorphism and hence $\phi\in \hat{A}$. If $A$ is unital and all $\phi_n$ are non-zero, then $\phi_n(1)=1$ for all $n\in\N$ by Remark \ref{Rmk: Characters unital} and especially $\phi(1)=1$, so $\phi$ is non-zero as well - hence in that case, $\hat{A}$ is already compact.
\end{proof}

\subsection{The Gelfand transform}
\begin{definition}
	We define the \textit{Gelfand transform} to be the map
	\begin{align*}
		\Gamma\colon A&\to \cC_0(\hat{A})\\
		a&\mapsto (\chi\mapsto \chi(a)).
	\end{align*}
\end{definition}
\begin{remark}
	This is well defined: $\Gamma(a)$ is bounded as
	\begin{align*}
		\abs{\Gamma(a)(\chi)}=\abs{\chi(a)}\leq \norm{\chi}\norm{a}\leq \norm{a}.
	\end{align*}
	In other words, $\Gamma(a)\in \cC_b(\hat{A})$. Now, let $\eps>0$ and pick a sequence $(\chi_n)_{n\in\N}$ of elements in $\hat{A}\cup \{0\}$ with $\abs{\chi_n(a)}\leq \eps$ for all $n\in \N$ that converges to an element $\chi\in \hat{A}\cup\{0\}$. Then $\abs{\chi(a)}=\lim_{n\to\infty}\abs{\chi_n(a)}\leq \eps$, hence the set
	\begin{align*}
		K_{a,\eps}\coloneqq \{\chi\in \hat{A}\cup \{0\}\mid \abs{\chi(a)}\leq \eps\}
	\end{align*}
	is a closed subspace of the compact space $\hat{A}\cup\{0\}$ and thus compact. But further
	\begin{align*}
		\Gamma(a)\inv((-\eps,\eps))\subseteq K_{a,\eps},
	\end{align*}
	so $\Gamma(a)\in \cC_0(\hat{A})$.
\end{remark}
\begin{theorem}\label{Thm: Gelfand transform is norm-decreasing algebra homomorphism}
	$\Gamma$ is a norm-decreasing algebra homomorphism and $\Gamma(A)$ separates points in $\hat{A}$.
\end{theorem}
\begin{proof}
	We have 
	$$
	\norm{\Gamma(a)}=\sup_{\chi\in \hat{A}}|\chi(a)|\leq \sup_{\chi\in \hat{A}}\norm{\chi}\norm{a}\leq \norm{a},
	$$
	so $\Gamma$ is indeed norm-decreasing. Furthermore,
	$$
	\Gamma(ab)=(\chi\mapsto \chi(ab))=(\chi\mapsto \chi(a))(\chi\mapsto\chi(b))=\Gamma(a)\Gamma(b)
	$$
	and an analogous argument shows $\C$-linearity, so $\Gamma$ is an algebra homomorphism. Lastly, for $\phi\neq\phi'\in \hat{A}$, there is some $a\in A$ with
	$$\Gamma(a)(\chi)=\chi(a)\neq \chi'(a)=\Gamma(a)(\chi'),$$
	hence $\Gamma(A)$ separates points in $\hat{A}$.
\end{proof}
\begin{example}
	We will see in \todo{reference} that for $A=\cC_0(X)$ for a locally compact topological space $X$, the Gelfand transform $\Gamma\colon A\to \cC_0(\hat{A})$ is an isomorphism.
\end{example}
\begin{proposition}
	If $A$ is unital and $a\in A$, we have $\sp(a)=\Im(\Gamma(a))$. If $A$ is not unital, $\widetilde{\sp}(a)=\Im(\Gamma(a))\cup\{0\}$, where $\widetilde{\sp}(a)\coloneqq \sp{((a,0))}$ in $\tilde{A}$.
\end{proposition}
\begin{proof}
	Assume first that $A$ is unital. We start by showing $\Im(\Gamma(a))\subseteq \sp(a)$. Let $\chi\in\hat{A}$. Then
	\begin{align*}
		\chi(\chi(a)\cdot 1-a)=\chi(a)\chi(1)-\chi(a)=0\not\in\C^{\times},
	\end{align*}
	so $\chi(a)\cdot 1-a$ is not invertible and hence $\Gamma(a)(\chi)=\chi(a)\in \sp(a)$.
	
	For the opposite inclusion, let $\lambda\in \sp(a)$. We aim to construct a character $\chi_{\lambda}:A\to\C$ that sends $a$ to $\lambda$.
	\todo{Think about this}
\end{proof}
\begin{corollary}\label{Cor: Norm of Gelfand transform is spectral radius}
	For $A$ unital and $a\in A$ we have $\norm{\Gamma(a)}=r(a)$. In particular, $\ker \Gamma=\{a\in A|r(a)=0\}$.
\end{corollary}
\subsection{$C^*$-algebras}
\begin{definition}
	An \textit{involution} on a Banach algebra $A$ is a map $(-)^*:A\to A$ such that:
	\begin{itemize}
		\item For all $\lambda\in \C, a,b\in A$ we have $(\lambda a+b)^*=\conj{\lambda}a^*+b^*$.
		\item For $a,b\in A$ we have $(ab)^*=b^*a^*$.
		\item For all $a\in A$ we have $a^{**}=a$.
	\end{itemize}
\end{definition}
\begin{remark}
	One can, more generally, define an involution on a (non-unital) ring $R$ as a (non-unital) ring automorphism $(-)^*:R\to R\op$ with $((-)^*)\inv = ((-)^*)\op$, where $R\op$ is the ring with reversed multiplication. For such a ring with involution $R$, an involution on an $R$-algebra $A$ is an involution $(-)^*$ on $A$ making
	\begin{center}
		\begin{tikzcd}
			R\ar[d,"(-)^*"]\ar[r] & A\ar[d,"(-)^*"]\\
			R\ar[r] & A
		\end{tikzcd}
	\end{center}
	commute. An involution on a Banach algebra is then just an involution of the $\C$-algebra, where the involution on $\C$ is $\conj{(-)}$. This approach works in any category with a $C_2$-action.
\end{remark}
\begin{definition}
	A $C^*$-algebra is a Banach algebra $(A,\norm{.})$ with involution $(-)^*$ such that $\norm{a^*a}=\norm{a}^2$ for all $a\in A$.
\end{definition}
\begin{lemma}
	Let $A$ be a $C^*$-algebra and $a\in A$. Then $\norm{a^*}=\norm{a}$.
\end{lemma}
\begin{proof}
	For $a=0$ we have $a^*=\conj{0}=0$, hence $\norm{a}=\norm{a^*}$ is clear. Otherwise, we have $\norm{a}>0$ and
	\begin{align*}
		\norm{a}^2=\norm{a^*a}\leq \norm{a^*}\norm{a},
	\end{align*}
	thus dividing by $\norm{a}$ yields $\norm{a}\leq \norm{a^*}$. Applying the same result to $a^*$, we get
	\begin{align*}
		\norm{a}\leq\norm{a^*}\leq\norm{a^{**}}=\norm{a},
	\end{align*}
	so $\norm{a}=\norm{a^*}$.
\end{proof}
\begin{example}
	The following Banach algebras with involution are $C^*$-algebras:
	\begin{itemize}
		\item $\cC_b(X)$ and $\cC_0(X)$ with (pointwise) complex conjugation.
		\item For any Hilbert space $\cH$, the algebra $\cB(\cH)$ of bounded operators on $\cH$ with involution induced by $\langle T^*v|w\rangle=\langle v|Tw\rangle$.
	\end{itemize}
\end{example}
\begin{proof}
	We have already seen that $\cC_b(X)$ and $\cC_0(X)$ are commutative Banach algebras. Pointwise conjugation being an involution is a simple check. Lastly,
	\begin{align*}
		\norm{a^*a}=\sup_{x\in X} \abs{a(x)\conj{a(x)}}=\sup_{x\in X}\abs{a(x)}^2=\norm{a}^2,
	\end{align*}
	so the given algebra with involution is indeed a $C^*$-algebra.
	
	We have also already seen that $\cB(\cH)$ is a Banach algebra. The axioms for an involution are again a straightforward check. Further, we have
	\begin{align*}
		\norm{T^*T}&=\sup_{\substack{v\in V\\\norm{v}=1}} \norm{T^*Tv}\\
			&=\sup_{\substack{v\in V\\\norm{v}=1}} \abs{\langle T^*Tv|T^*Tv\rangle}^{\frac{1}{2}}\\
			&=\sup_{\substack{v\in V\\\norm{v}=1}} \abs{\langle TTv|TTv\rangle}^{\frac{1}{2}}\\
			&=\sup_{\substack{v\in V\\\norm{v}=1}} \norm{TTv}\\
			&=\norm{TT}
	\end{align*}
	hence $\cB(\cH)$ is a $C^*$-algebra.
\end{proof}
\begin{remark}
	Let $A$ be a non-unital $C^*$-algebra. We can consider the unitization $\tilde{A}$ of the underlying Banach algebra: This is a unital Banach algebra with norm $\norm{(a,\lambda)}=\norm{a},\abs{\lambda}$, but with that norm it does not neccessarily admit the structure of a $C^*$-algebra compatible with that of $A$ (\weaktodo{Example for this phenomenon}). Hence we have to provide a different norm on $\tilde{A}$ if we aim to construct a unitization for $C^*$-algebras.
\end{remark}
\begin{definition}\label{Def: Unitization of C^*-algebra}
	Let $A$ be a $C^*$-algebra and consider the embedding
	\begin{align*}
		L\colon \tilde{A}&\to \cB(A)\\
		(a,\lambda)&\mapsto (b\mapsto ab+\lambda b).
	\end{align*}
	We will see in Lemma \ref{Lemma: Inclusion of unitization in bounded endomorphisms injective} that $L$ is an injective algebra homomorphism, hence we define the norm and $(-)^*$ on $\tilde{A}\cong L(\tilde{A})\subseteq \cB(A)$ to be the ones inherited from $\cB(A)$. We refer to this $C^*$-algebra as the \textit{unitization} of $A$.
\end{definition}
\begin{lemma}\label{Lemma: Inclusion of unitization in bounded endomorphisms injective}
	Let $A$ be a $C^*$-algebra. The map $L\colon \tilde{A}\to\cB(A)$ from Definition \ref{Def: Unitization of C^*-algebra} is an injective algebra homomorphism. Further, $\tilde{A}\cong L(\tilde A)$ with the norm and $(-)^*$ inherited from $\cB(A)$ forms a unital $C^*$-algebra and $A\subseteq \tilde{A}$ is a $C^*$-subalgebra.
\end{lemma}
\weaktodo{Prove at least parts of this}
\begin{note}
	The norm on $\tilde{A}$ is $\norm{(a,\lambda)}=\sup_{b\in A}\norm{ab+\lambda b}$, the involution is $(a,\lambda)^*=(a^*,\conj{\lambda})$.
\end{note}
\weaktodo{Note on the adjunction}
\begin{definition}
	Let $A$ be a $C^*$-algebra and $a\in A$. We say that $a$ is
	\begin{itemize}
		\item \textit{self-adjoint} if $a^*=a$,
		\item \textit{unitary} if $a^*a=aa^*=1$,
		\item \textit{normal} if $a^*a=aa^*$,
		\item \textit{positive} if $a=b^*b$ for some $b\in A$.
	\end{itemize}
\end{definition}
\begin{definition}
	For a (not neccessarily unital) $C^*$-algebra $A$ and $a\in A$, we define the spectral radius of $a$ to be $r(a)\coloneqq r((a,0))$, the spectral radius of $(a,0)\in \tilde{A}$.
\end{definition}
\begin{proposition}\label{Lemma: Norm of powers of normal elements}
	For a $C^*$-algebra $A$, $a\in A$ normal and $n\in \N$, we have $\norm{a^n}=\norm{a}^n$.
\end{proposition}
\begin{proof}
	Using normality and $(aa)^*=a^*a^*$ for the second equation, we have
	\begin{align*}
		\norm{a^2}&=\norm{a^2\left(a^2\right)^*}^{\frac{1}{2}}\\
		&=\norm{(a^*a)(a^*a)^*}^{\frac{1}{2}}\\
		&=\norm{a^*a}\\
		&=\norm{a}^2.
	\end{align*}
	Inductively, $\norm{a^{2^n}}=\norm{a}^{2^n}$. Further, by submultiplicativity, for any $n\in \N$ we have $\norm{a^n}\leq \norm{a}^n$. But then, letting $k\in \N$ with $2^k>n$, we get
	\begin{align*}
		\norm{a}^n\norm{a^{2^k-n}}&\leq \norm{a}^n\norm{a}^{2^k-n}\\
		&=\norm{a}^{2^k}\\
		&=\norm{a^{2^k}}\\
		&\leq \norm{a^n}\norm{a^{2^k-n}}.
	\end{align*}
	Now, if $a=0$, the statement to be proven is trivial, otherwise we get
	\begin{align*}
		\norm{a}^n\leq \norm{a^n}\leq\norm{a}^n,
	\end{align*}
	which yields the desired equality.
\end{proof}
\begin{corollary}
	For a $C^*$-algebra $A$ and $a\in A$ normal, we have $r(a)=\norm{a}$.
\end{corollary}
\begin{proof}
	Let $\alpha=(a,0)\in \tilde{A}$. As $a$ was normal, so is $\alpha$. By Theorem \ref{Thm: Spectral Theorem} and Lemma \ref{Lemma: Norm of powers of normal elements}, we have
	\begin{align*}
		r(a)=r(\alpha)=\lim_{n\to\infty}\norm{\alpha^n}^{\frac{1}{n}}=\lim_{n\to\infty}\left(\norm{\alpha}^n\right)^{\frac{1}{n}}=\norm{\alpha}.
	\end{align*}
\end{proof}
\begin{corollary}\label{Cor: Spectral radius of normal elements}
	For a $C^*$-algebra $A$ and $a\in A$, we have $\norm{a}=r(a^*a)^{\frac{1}{2}}$. In particular, the norm of a $C^*$-algebra is already uniquely determined by the $\C$-algebra structure.
\end{corollary}
\begin{proof}
	$a^*a$ is self-adjoint, hence normal, so by Corollary \ref{Cor: Spectral radius of normal elements} we get
	\begin{align*}
		\norm{a}=\norm{a^*a}^{\frac{1}{2}}=r(a^*a)^{\frac{1}{2}}.
	\end{align*}
\end{proof}
\begin{corollary}\label{Cor: Gelfand transform injective}
	If $A$ is a commutative $C^*$-algebra, the Gelfand transform $\Gamma$ is in isometry.
\end{corollary}
\begin{proof}
	By Corollary \ref{Cor: Norm of Gelfand transform is spectral radius} and Corollary \ref{Cor: Spectral radius of normal elements}, we have $\norm{\Gamma(a)}=r(a)=\norm{a}$ for every normal $a\in A$. But as $A$ is commutative, every element is normal.
\end{proof}
\begin{proposition}\label{Prop: Spectrum of self-adjoint and unitary elements}
	Let $A$ be a unital $C^*$-algebra and $a\in A$. If $a$ is self-adjoint, $\sp(a)\subseteq\R$. If $a$ is unitary, $\sp(a)\subseteq U(1)=S^1$.
\end{proposition}
\todo{Proof}
\begin{lemma}\label{Lemma: Real and imaginary part in C^*-algebra}
	For a $C^*$-algebra $A$ and $a\in A$, we can find self-adjoint elements $b,c\in A$ with $a=b+ic$.
\end{lemma}
\begin{proof}
	Let 
	\begin{align*}
		b\coloneqq \frac{a+a^*}{2}
	\end{align*}
	and
	\begin{align*}
		c\coloneqq \frac{i(a^*-a)}{2}.
	\end{align*}
	Then $b^*=b$, $c^*=c$ and $b+ic=a$.
\end{proof}
\begin{lemma}\label{Lemma: Characters are *-homomorphisms}
	For a $C^*$-algebra $A$, each character $\chi:A\to \C$ is already a $C^*$-algebra homomorphism.
\end{lemma}
\begin{proof}
	Let $\chi\in \hat{A}$ and $a\in A$. We write $a=b+ic$ with $b,c\in A$ self-adjoint (Lemma \ref{Lemma: Real and imaginary part in C^*-algebra}). Note first that $\chi(b)\in\sp(b)\subseteq \R$ by Proposition \ref{Prop: Spectrum of self-adjoint and unitary elements}, and the same holds for $\chi(c)$. But then
	\begin{align*}
		\chi(a^*)&=\chi(b^*-ic^*)\\
		&=\chi(b-ic)\\
		&=\chi(b)-i\chi(c)\\
		&=\conj{\chi(b)+i\chi(c)}\\
		&=\conj{\chi(b+ic)}\\
		&=\conj{\chi(a)}
	\end{align*}
	as desired.
\end{proof}
We use the following input from functional analysis:
\begin{theorem}[Stone-Weierstrass]\label{Thm: Stone-Weierstrass}
	If $X$ is a locally compact Hausdorff space and $B\subseteq\cC_0(X)$ a nowhere vanishing, self-adjoint algebra which separates points, then $B\subseteq\cC_0(X)$ is dense.
\end{theorem}
Using this, we can show:
\begin{theorem}[Gelfand-Naimark I]
	Let $A$ be a commutative $C^*$-algebra. Then the Gelfand transform $\Gamma$ is an isomorphism of $C^*$-algebras.
\end{theorem}
\begin{proof}
	We have already seen that $\Gamma$ is an algebra homomorphism and $\Gamma(a)$ separates points in $\hat{A}$ for all $a\in A$ (Theorem \ref{Thm: Gelfand transform is norm-decreasing algebra homomorphism}). Further, $\Gamma$ is a morphism of $C^*$-algebras: 
	Using Lemma \ref{Lemma: Characters are *-homomorphisms}, we get
	\begin{align*}
		\Gamma(a^*)=(\chi\mapsto\chi(a^*))=(\chi\mapsto \conj{\chi(a)})=\Gamma(a)^*.
	\end{align*}
	We have further already seen that $\Gamma$ is an isometry (Corollary \ref{Cor: Gelfand transform injective}), hence it is injective and, as $A$ is complete, the image of $\Gamma$ is closed, so it suffices to show that this image is dense. For that, we apply Theorem \ref{Thm: Stone-Weierstrass}:
	\begin{itemize}
		\item $\Gamma(A)$ is nowhere vanishing as for each $\chi\in \hat{A}$ we have $\chi\neq 0$, so there is some $a\in A$ with $\Gamma(a)(\chi)=\chi(a)\neq 0$.
		\item $\Gamma(A)$ is self-adjoint, as $\Gamma(a)^*=\Gamma(a^*)$.
		\item $\Gamma(A)$ seperates points by Theorem \ref{Thm: Gelfand transform is norm-decreasing algebra homomorphism}.
	\end{itemize}
	So the conditions for Stone-Weierstrass are met and $\Gamma(A)\subseteq \cC_0(\hat{A})$ is dense, which was left to show.
\end{proof}
\begin{proposition}\label{Prop: Spectrum of algebra of continuous functions on X is X}
	Let $X$ be a compact Hausdorff space. Then the map
	\begin{align*}
		\eps: X&\to \widehat{\cC_0(X)}\\
		x&\mapsto (\phi\mapsto \phi(x))
	\end{align*}
	is an isomorphism.
\end{proposition}
\begin{proof}
	\leavevmode
	\begin{itemize}
		\item First, note that $\eps$ is well-defined: As $X$ is compact, $\const_1\in \cC_0(X)$ and $\eps(x)(\const_1)=\const_1(x)=1$, hence $\eps(x)$ is nowhere vanishing. Further, it is easy to see that $\eps(x)$ is an algebra homomorphism, hence a character.
		\item Continuity of $\eps$ turns out to be rather subtle: It is easy to show sequential continuity, but that does not neccessarily imply continuity. Instead, one can show that net continuity implies continuity, and that a net $(T_\lambda)_{\lambda\in I}$ in the weak-$*$ topology converges if and only if it converges pointwise. Then, for a net $(x_\lambda)_{\lambda\in I}$ in $X$, as every $\phi\in \cC_0(X)$ is continuous, we have
		\begin{align*}
			(\eps(x_\lambda)(\phi))_{\lambda\in I}=(\phi(x_\lambda))_{\lambda}\to \phi(x)=\eps(x)(\phi)
		\end{align*}
		Hence, $\eps(x_\lambda)$ is a pointwise converging net, thus by the aforementioned theorem already a weak-$*$ convergent net. So $\eps$ is continuous.
		\item $\eps$ is injective: Let $x\neq y\in X$ with $\eps(x)=\eps(y)$. As $\{x\}$ and $\{y\}$ are closed, by Urysohns lemma there is some continuous $\phi:X\to [0,1]\into \C$ with $\phi(x)\neq \phi(y)$, hence $\eps(x)(\phi)\neq \eps(y)(\phi)$.
		\item $\eps$ is surjective: Let $\chi:\cC_0(X)\to \C$ be a character. We aim to apply Stone-Weierstrass to $\ker\chi$. For $\phi\in \cC_0(X)$, we have 
		\begin{align*}
			\chi(\chi(\phi)\cdot 1-\phi)=0,
		\end{align*}
		so $\chi(\phi)\cdot 1-\phi\in \ker \chi$. Now, if $x\neq y\in X$, there is some $\phi\in \cC_0(X)$ with $\phi(x)\neq \phi(y)$, hence
		\begin{align*}
			\phi(x)-\chi(\phi)\neq \phi(y)-\chi(\phi).
		\end{align*}
		Thus, $\ker \chi$ separates points. Further, $\ker \chi$ is the kernel of a $C^*$-algebra homomorphism, as we have seen in Lemma \ref{Lemma: Characters are *-homomorphisms}, hence a self-adjoint algebra. Therefore, if $\ker \chi$ would separate points, it would be dense in $\widehat{\cC_0(X)}$, but it is also closed, hence that would already imply $\ker \chi=\widehat{\cC_0(X)}$, a contradiction. Thus, $\ker\chi$ does not separate points, so there is some $x\in X$ with $\chi(\phi)-\phi(x)=0$ for all $\phi\in\cC_0(X)$. But this just means $\chi(\phi)=\phi(x)=\eps(x)(\phi)$, so $\chi=\eps(x)$.
	\end{itemize}
	To conclude, note that this yielded a continuous bijection $X\to\widehat{\cC_0(X)}$, where $X$ is compact Hausdorff, so this is already a homeomorphism.
\end{proof}
\begin{proposition}
	The map $\eps: X\to \widehat{\cC_0(X)}$ from Proposition \ref{Prop: Spectrum of algebra of continuous functions on X is X} is already an equivalence if $X$ is locally compact Hausdorff.
\end{proposition}
\begin{proof}[Proof Sketch]
	Let $\tilde{X}=X\cup \{\infty\}$ denote the one-point compactification. Then, a map $\phi\in \cC_0(\tilde{X})$ is of the form $\const_{\phi(\infty)}+\phi'$, where $\phi'(\infty)=0$ and $\phi'\mid_X\in \cC_0(X)$. This yields
	\begin{align*}
		\cC_0(\tilde{X})\cong \cC_0(X)\oplus\C
	\end{align*}
	and one can check that $\cC_0(\tilde{X})\cong \widetilde{\cC_0(X)}$, the unitization of the $C^*$-algebra $\cC_0(X)$. Further, for any $C^*$-algebra one can check $\hat{\tilde{A}}=\hat{A}\cup\{\chi_\infty\}$ , where $\chi_\infty:\tilde{A}\cong A\oplus\C\to \C$ is the projection onto the second component. But then, applying Proposition \ref{Prop: Spectrum of algebra of continuous functions on X is X} to $\tilde{X}$ finishes the proof.
\end{proof}
\begin{remark}
	One can check that the assignments $\cC_0:\Top^{lc,H}\to C^*\Alg^{\comm}$ and $\widehat{(-)}:C^*\Alg^{\comm}\to \Top^{lc,H}$ are contravariant functors, where $\Top^{lc,H}$ is the category of locally compact Hausdorff spaces and $C^*\Alg^{\comm}$ that of commutative $C^*$-algebras. Further, the Gelfand transform and $\eps$ from Proposition \ref{Prop: Spectrum of algebra of continuous functions on X is X} are natural, hence $\cC_0$ and $\widehat{(-)}$ form a contravariant equivalence between locally compact Hausdorff spaces and commutative $C^*$-algebras. Under this equivalence, one-point compactification corresponds to unitization.
\end{remark}
\end{document}
